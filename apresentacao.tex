\documentclass{beamer}
\usepackage[brazil]{babel}
%\usepackage[latin1]{inputenc}
\usepackage[utf8x]{inputenc} 
%\usepackage[all]{xy}

\usetheme{Darmstadt}

\title{Guia de estilo para programar em Ruby}

\author{Rodrigo L. M. Flores \\ \url{mail@rodrigoflores.org}}

\institute{Projeto Archlinux-BR}

\begin{document}

\date{\today}

\frame{\titlepage}

\begin{frame}
    \frametitle{IMHO...}
    \begin{block}{Coisas a serem consideradas em um Guia de Estilo}
        \begin{itemize}
            \item Guias de estilo são como um buffet e não como uma escritura sagrada
            \item Mas segui-los como um escritura sagrada faz nosso código ser melhor entendido pelos outros...
            \item Guias de estilo ajudam a manter um código mais bonito, mas não são a única coisa que mantém um código bonito
            \item Mesmo que um guia de estilo não seja seguido a risca, todo o código deve seguir as mesmas diretrizes mesmo guia
        \end{itemize}
    \end{block}
\end{frame}

\begin{frame}
    \frametitle{Organização do código}
    \begin{enumerate}
        \item Cabeçalho com nome do autor e descrição do programa
        \item Require's
        \item Include's
        \item Definições de classes e módulos
        \item Programa principal
        \item Código de teste
    \end{enumerate}
\end{frame}

% Sugestão de cabeçalho

\begin{frame}[fragile]
    \frametitle{Sugestão de cabeçalho}
    \begin{block}{Sugestão de cabeçalho ?}
        \begin{verbatim}
=begin
  * Name:
  * Description   
  * Author:
  * Date:
  * License:
=end
        \end{verbatim}


    \end{block}
\end{frame}



\begin{frame}
    \frametitle{Require x Include}
    \begin{itemize}
        \item[Require] faz o que o ``include'' de outras linguagens faz: roda o outro arquivo
        \item[Include] pega todos os métodos de outro módulo e os inclui no módulo atual
    \end{itemize}
\end{frame}

\begin{frame}[fragile]
    \frametitle{Warnings}
    \begin{block}{Warnings em Ruby}
        \begin{itemize}
            \item Flag \verb@-w@ no interpretador (colocar no shebang também)
        \end{itemize}
    \end{block}
\end{frame}

\begin{frame}[fragile]
    \frametitle{Variáveis globais}
    \begin{block}{Quando usar ?}
        Para evitar problemas com conflitos de namespaces e referências acidentais
        \begin{itemize}
            \item One-liners
            \item Variáveis globais pré-definidas (\verb#$. $! $?#)
        \end{itemize}
    \end{block}
    \begin{block}{IMHO...}
        Cabe a cada um utilizar quando achar necessário. Mas tomar cuidado quando usar.
    \end{block}
\end{frame}

\begin{frame}[fragile]
    \frametitle{Nome de coisas}
    \begin{itemize}
        \centering
        \item[Classes] \verb#CamelCase#, substantivos
        \item[Módulos] \verb#CamelCase#, não é fácil recomendar
        \item[Constantes] \verb#SCREAMING_SNAKE_CASE#, substantivos
        \item[Siglas] \verb#SCREAMINGCASE# Exemplo: \verb#HTTP XML IME#, 
        \item[Nomes de métodos] \verb#snake_case#, se possível deve ser um verbo
        \item[Variáveis] \verb#snake_case#, substantivos (se a variável for local, pode abreviar)
    \end{itemize}
\end{frame}

\begin{frame}[fragile]
    \frametitle{Inicialização de (Hash|Array|String)}
    \begin{block}{Dois jeitos possíveis}
        Escolher (e manter) um deles: \verb#$1.new # ou \verb#{} [] ""#
    \end{block}
\end{frame}

\begin{frame}[fragile]
    \frametitle{Métodos booleanos}
    \begin{block}{Devem terminar em ``?''}
        Exemplos:\\
        \verb#Math.positive? #\\
        \verb#Stack.empty?   #\\
    \end{block}
\end{frame}

\begin{frame}[fragile]
    \frametitle{Ponto e vírgula}
    \begin{block}{Não usar isso!}
        Exemplos:\\
        \verb#($stderr.puts "Error!"; exit 1) unless x == 2#
    \end{block}
\end{frame}

\begin{frame}[fragile]
    \frametitle{Associações múltiplas}
    \begin{block}{Use, mas tome cuidado com referências}
        Exemplo de pegadinha :\\
        \begin{verbatim}
irb(main):001:0> foo = bar = Array.new
=> []
irb(main):002:0> foo
=> []
irb(main):003:0> bar
=> []
irb(main):004:0> foo << "baz"
=> ["baz"]
irb(main):005:0> foo
=> ["baz"]
irb(main):006:0> bar
=> ["baz"]
        \end{verbatim}
    \end{block}
\end{frame}

% Tamanho da linha

\begin{frame}[fragile]
    \frametitle{Tamanho da linha}
    \begin{block}{Máximo}
        $80$ ou $79$.
    \end{block}
    \begin{block}{E se não couber em uma linha o que eu preciso fazer ?}
        Se o interpretador achar que uma linha não faz sentido, ele tentará achar uma continuação na linha de baixo.
    \end{block}
\end{frame}


% Recuo (aka i(n?)dentação)
\begin{frame}
    \frametitle{Recuo (aka i(n?)dentação)}
    \begin{block}{2, 4, 8 ? Espaços, tabulações?}
        2 espaços. Mas isso não é uma consenso!
    \end{block}
\end{frame}


% Linhas em branco

\begin{frame}
    \frametitle{Linhas em branco}
    \begin{block}{Onde quantas linhas em branco eu deixo?}
        \begin{itemize}   
            \item Deixar $2$ linhas em branco entre cada definição de classe e módulo
            \item Deixar $1$ linha em branco entre cada definição de método
        \end{itemize}
    \end{block}
\end{frame}


% Espaços em branco

\begin{frame}[fragile]
    \frametitle{Espaços em branco}
    \begin{block}{Quando usar ?}
        Atribuição:\\
        \verb#a = 1# e não \verb#a=1#
    \end{block}
    \begin{block}{Quando não usar ?}
        Valores padrões em argumentos:\\
        \verb#def foo(a, b=1)# e não \verb#def foo(a, b = 1)#\\
        Não separar lista de argumentos da chamada do método:\\
        \verb#foo.bar(a)# e não \verb#foo.bar (a)#
    \end{block}
\end{frame}

% Espaços em branco

\begin{frame}[fragile]
    \frametitle{``Ou igual'' }
    \begin{block}{$||=$}
        Use e abuse!\\
        \verb#x ||= 2#\\
        Mesma coisa que:\\
        \verb#x = 2 if x.nil?#\\
    \end{block}
\end{frame}

\begin{frame}
  \frametitle{Sites úteis}
  \begin{itemize}
    \item Guia de estilo (mais prolixo) : \url{http://www.caliban.org/ruby/rubyguide.shtml} 
    \item Guia de estilo (mais resumido) : \url{http://github.com/chneukirchen/styleguide/tree/master} 
    \item Dicas de Leiaute do Prof. Feofiloff: \url{http://www.ime.usp.br/~pf/algoritmos/aulas/layout.html} 
  \end{itemize}
\end{frame}

\begin{frame}
    \frametitle{Dúvidas}
    \begin{block}{Contato :}
        \begin{itemize}
            \centering
            \item[E-mail] flores@archlinux-br.org        
            \item[XMPP]  im@rodrigoflores.org        
            \item[Site]  \url{http://rodrigoflores.org}
            \item[Site do arch-br]  \url{http://flores.archlinux-br.org}
            \item[Blog]  \url{http://blog.rodrigoflores.org}        
            \item[Twitter] rodrigoflores        
            \item[Identi.ca] rodrigoflores        
            \item[Jaiku] flores        
        \end{itemize}
    \end{block}

\end{frame}

\end{document}


% vim:set ts=4 expandtab:
